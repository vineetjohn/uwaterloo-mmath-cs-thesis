% uWaterloo Thesis Template for LaTeX 
% Last Updated June 14, 2017 by Stephen Carr, IST Client Services
% FOR ASSISTANCE, please send mail to rt-IST-CSmathsci@ist.uwaterloo.ca

% Effective October 2006, the University of Waterloo 
% requires electronic thesis submission. See the uWaterloo thesis regulations at
% https://uwaterloo.ca/graduate-studies/thesis.

% DON'T FORGET TO ADD YOUR OWN NAME AND TITLE in the "hyperref" package
% configuration below. THIS INFORMATION GETS EMBEDDED IN THE PDF FINAL PDF DOCUMENT.
% You can view the information if you view Properties of the PDF document.

% Many faculties/departments also require one or more printed
% copies. This template attempts to satisfy both types of output. 
% It is based on the standard "book" document class which provides all necessary 
% sectioning structures and allows multi-part theses.

% DISCLAIMER
% To the best of our knowledge, this template satisfies the current uWaterloo requirements.
% However, it is your responsibility to assure that you have met all 
% requirements of the University and your particular department.
% Many thanks for the feedback from many graduates that assisted the development of this template.

% -----------------------------------------------------------------------

% By default, output is produced that is geared toward generating a PDF 
% version optimized for viewing on an electronic display, including 
% hyperlinks within the PDF.
 
% E.g. to process a thesis called "mythesis.tex" based on this template, run:

% pdflatex mythesis	-- first pass of the pdflatex processor
% bibtex mythesis	-- generates bibliography from .bib data file(s)
% makeindex         -- should be run only if an index is used 
% pdflatex mythesis	-- fixes numbering in cross-references, bibliographic references, glossaries, index, etc.
% pdflatex mythesis	-- fixes numbering in cross-references, bibliographic references, glossaries, index, etc.

% If you use the recommended LaTeX editor, Texmaker, you would open the mythesis.tex
% file, then click the PDFLaTeX button. Then run BibTeX (under the Tools menu).
% Then click the PDFLaTeX button two more times. If you have an index as well,
% you'll need to run MakeIndex from the Tools menu as well, before running pdflatex
% the last two times.

% N.B. The "pdftex" program allows graphics in the following formats to be
% included with the "\includegraphics" command: PNG, PDF, JPEG, TIFF
% Tip 1: Generate your figures and photos in the size you want them to appear
% in your thesis, rather than scaling them with \includegraphics options.
% Tip 2: Any drawings you do should be in scalable vector graphic formats:
% SVG, PNG, WMF, EPS and then converted to PNG or PDF, so they are scalable in
% the final PDF as well.
% Tip 3: Photographs should be cropped and compressed so as not to be too large.

% To create a PDF output that is optimized for double-sided printing: 
%
% 1) comment-out the \documentclass statement in the preamble below, and
% un-comment the second \documentclass line.
%
% 2) change the value assigned below to the boolean variable
% "PrintVersion" from "false" to "true".

% --------------------- Start of Document Preamble -----------------------

% Specify the document class, default style attributes, and page dimensions
% For hyperlinked PDF, suitable for viewing on a computer, use this:
\documentclass[letterpaper,12pt,titlepage,oneside,final]{book}
 
% For PDF, suitable for double-sided printing, change the PrintVersion variable below
% to "true" and use this \documentclass line instead of the one above:
%\documentclass[letterpaper,12pt,titlepage,openright,twoside,final]{book}

% Some LaTeX commands I define for my own nomenclature.
% If you have to, it's better to change nomenclature once here than in a 
% million places throughout your thesis!
\newcommand{\package}[1]{\textbf{#1}} % package names in bold text
\newcommand{\cmmd}[1]{\textbackslash\texttt{#1}} % command name in tt font 
\newcommand{\href}[1]{#1} % does nothing, but defines the command so the
\newcommand{\rmx}{\mathrm x}
    % print-optimized version will ignore \href tags (redefined by hyperref pkg).
%\newcommand{\texorpdfstring}[2]{#1} % does nothing, but defines the command
% Anything defined here may be redefined by packages added below...

% This package allows if-then-else control structures.
\usepackage{ifthen}
\newboolean{PrintVersion}
\setboolean{PrintVersion}{false} 
% CHANGE THIS VALUE TO "true" as necessary, to improve printed results for hard copies
% by overriding some options of the hyperref package below.

%\usepackage{nomencl} % For a nomenclature (optional; available from ctan.org)
\usepackage{amsmath,amssymb,amstext} % Lots of math symbols and environments
\usepackage[pdftex]{graphicx} % For including graphics N.B. pdftex graphics driver 
\usepackage{natbib} % For better bibliography formatting
\usepackage[dvipsnames]{xcolor} % Additional colors
\usepackage{multirow}
\usepackage{bm}


% Hyperlinks make it very easy to navigate an electronic document.
% In addition, this is where you should specify the thesis title
% and author as they appear in the properties of the PDF document.
% Use the "hyperref" package 
% N.B. HYPERREF MUST BE THE LAST PACKAGE LOADED; ADD ADDITIONAL PKGS ABOVE
\usepackage[pdftex,pagebackref=false]{hyperref} % with basic options
		% N.B. pagebackref=true provides links back from the References to the body text. This can cause trouble for printing.
\hypersetup{
    plainpages=false,       % needed if Roman numbers in frontpages
    unicode=false,          % non-Latin characters in Acrobat’s bookmarks
    pdftoolbar=true,        % show Acrobat’s toolbar?
    pdfmenubar=true,        % show Acrobat’s menu?
    pdffitwindow=false,     % window fit to page when opened
    pdfstartview={FitH},    % fits the width of the page to the window
    pdftitle={Stylistic\ Variation\ in\ Linguistics\ using\ Disentangled\ Latent\ Spaces},    % title: CHANGE THIS TEXT!
    pdfauthor={Vineet John},    % author: CHANGE THIS TEXT! and uncomment this line
    pdfsubject={Natural Language Processing},  % subject: CHANGE THIS TEXT! and uncomment this line
   	pdfkeywords={nlp} {deep-learning} {linguistic}, % list of keywords, and uncomment this line if desired
    pdfnewwindow=true,      % links in new window
    colorlinks=true,        % false: boxed links; true: colored links
    linkcolor=blue,         % color of internal links
    citecolor=OliveGreen,        % color of links to bibliography
    filecolor=magenta,      % color of file links
    urlcolor=cyan           % color of external links
}
\ifthenelse{\boolean{PrintVersion}}{   % for improved print quality, change some hyperref options
\hypersetup{	% override some previously defined hyperref options
%    colorlinks,%
    citecolor=black,%
    filecolor=black,%
    linkcolor=black,%
    urlcolor=black}
}{} % end of ifthenelse (no else)

\usepackage[automake,toc,abbreviations]{glossaries-extra} % Exception to the rule of hyperref being the last add-on package
% If glossaries-extra is not in your LaTeX distribution, get it from CTAN (http://ctan.org/pkg/glossaries-extra), 
% although it's supposed to be in both the TeX Live and MikTeX distributions. There are also documentation and 
% installation instructions there.

% Setting up the page margins...
% uWaterloo thesis requirements specify a minimum of 1 inch (72pt) margin at the
% top, bottom, and outside page edges and a 1.125 in. (81pt) gutter
% margin (on binding side). While this is not an issue for electronic
% viewing, a PDF may be printed, and so we have the same page layout for
% both printed and electronic versions, we leave the gutter margin in.
% Set margins to minimum permitted by uWaterloo thesis regulations:
\setlength{\marginparwidth}{0pt} % width of margin notes
% N.B. If margin notes are used, you must adjust \textwidth, \marginparwidth
% and \marginparsep so that the space left between the margin notes and page
% edge is less than 15 mm (0.6 in.)
\setlength{\marginparsep}{0pt} % width of space between body text and margin notes
\setlength{\evensidemargin}{0.125in} % Adds 1/8 in. to binding side of all 
% even-numbered pages when the "twoside" printing option is selected
\setlength{\oddsidemargin}{0.125in} % Adds 1/8 in. to the left of all pages
% when "oneside" printing is selected, and to the left of all odd-numbered
% pages when "twoside" printing is selected
\setlength{\textwidth}{6.375in} % assuming US letter paper (8.5 in. x 11 in.) and 
% side margins as above
\raggedbottom

% The following statement specifies the amount of space between
% paragraphs. Other reasonable specifications are \bigskipamount and \smallskipamount.
\setlength{\parskip}{\medskipamount}

% The following statement controls the line spacing.  The default
% spacing corresponds to good typographic conventions and only slight
% changes (e.g., perhaps "1.2"), if any, should be made.
\renewcommand{\baselinestretch}{1} % this is the default line space setting

% By default, each chapter will start on a recto (right-hand side)
% page.  We also force each section of the front pages to start on 
% a recto page by inserting \cleardoublepage commands.
% In many cases, this will require that the verso page be
% blank and, while it should be counted, a page number should not be
% printed.  The following statements ensure a page number is not
% printed on an otherwise blank verso page.
\let\origdoublepage\cleardoublepage
\newcommand{\clearemptydoublepage}{%
  \clearpage{\pagestyle{empty}\origdoublepage}}
\let\cleardoublepage\clearemptydoublepage

% Define Glossary terms (This is properly done here, in the preamble. Could be \input{} from a file...)
% Main glossary entries -- definitions of relevant terminology
 
\makeglossaries

%======================================================================
%   L O G I C A L    D O C U M E N T -- the content of your thesis
%======================================================================
\begin{document}

% For a large document, it is a good idea to divide your thesis
% into several files, each one containing one chapter.
% To illustrate this idea, the "front pages" (i.e., title page,
% declaration, borrowers' page, abstract, acknowledgements,
% dedication, table of contents, list of tables, list of figures,
% nomenclature) are contained within the file "uw-ethesis-frontpgs.tex" which is
% included into the document by the following statement.
%----------------------------------------------------------------------
% FRONT MATERIAL
%----------------------------------------------------------------------
% T I T L E   P A G E
% -------------------
% Last updated June 14, 2017, by Stephen Carr, IST-Client Services
% The title page is counted as page `i' but we need to suppress the
% page number. Also, we don't want any headers or footers.
\pagestyle{empty}
\pagenumbering{roman}

% The contents of the title page are specified in the "titlepage"
% environment.
\begin{titlepage}
	\begin{center}
		\vspace*{1.0cm}

		\Huge
		{\bf Disentangled Latent Representation Learning for Stylistic Variation in Language Models}

		\vspace*{1.0cm}

		\normalsize
		by \\

		\vspace*{1.0cm}

		\Large
		Vineet John \\

		\vspace*{3.0cm}

		\normalsize
		A thesis \\
		presented to the University of Waterloo \\
		in fulfillment of the \\
		thesis requirement for the degree of \\
		Master of Mathematics \\
		in \\
		Computer Science \\

		\vspace*{2.0cm}

		Waterloo, Ontario, Canada, 2017 \\

		\vspace*{1.0cm}

		\copyright\ Vineet John 2017 \\
	\end{center}
\end{titlepage}

% The rest of the front pages should contain no headers and be numbered using Roman numerals starting with `ii'
\pagestyle{plain}
\setcounter{page}{2}

\cleardoublepage % Ends the current page and causes all figures and tables that have so far appeared in the input to be printed.
% In a two-sided printing style, it also makes the next page a right-hand (odd-numbered) page, producing a blank page if necessary.

% D E C L A R A T I O N   P A G E
% -------------------------------
% The following is a sample Declaration Page as provided by the GSO
% December 13th, 2006.  It is designed for an electronic thesis.
\noindent
I hereby declare that I am the sole author of this thesis. This is a true copy of the thesis, including any required final revisions, as accepted by my examiners.

\bigskip

\noindent
I understand that my thesis may be made electronically available to the public.

\cleardoublepage

% A B S T R A C T
% ---------------

\begin{center}\textbf{Abstract}\end{center}

This is the abstract.

\cleardoublepage

% A C K N O W L E D G E M E N T S
% -------------------------------

\begin{center}\textbf{Acknowledgements}\end{center}

These are the acknowledgements.

\cleardoublepage

% T A B L E   O F   C O N T E N T S
% ---------------------------------
\renewcommand\contentsname{Table of Contents}
\tableofcontents
\cleardoublepage
\phantomsection    % allows hyperref to link to the correct page

% L I S T   O F   T A B L E S
% ---------------------------
\addcontentsline{toc}{chapter}{List of Tables}
\listoftables
\cleardoublepage
\phantomsection		% allows hyperref to link to the correct page

% L I S T   O F   F I G U R E S
% -----------------------------
\addcontentsline{toc}{chapter}{List of Figures}
\listoffigures
\cleardoublepage
\phantomsection		% allows hyperref to link to the correct page

% GLOSSARIES (Lists of definitions, abbreviations, symbols, etc. provided by the glossaries-extra package)
% -----------------------------
% \printglossaries
% \cleardoublepage
% \phantomsection		% allows hyperref to link to the correct page

% Change page numbering back to Arabic numerals
\pagenumbering{arabic}


%----------------------------------------------------------------------
% MAIN BODY
%----------------------------------------------------------------------
% Because this is a short document, and to reduce the number of files
% needed for this template, the chapters are not separate
% documents as suggested above, but you get the idea. If they were
% separate documents, they would each start with the \chapter command, i.e, 
% do not contain \documentclass or \begin{document} and \end{document} commands.
%======================================================================
\chapter{Introduction}
%======================================================================


Natural Language Processing (NLP) is a sub-field of Artificial Intelligence (AI) that deals with the understanding and generation of human languages.

Recently many of the statistical NLP methods are giving way to neural models to parameterize more expressive models of language. This includes machine translation, dialogue modeling, abstract summarization, document classification etc.

The problem this thesis attempts to tackle is the neural disentanglement of style and content in text to enable conditioned generation of text. This is analogous to style transfer in computer vision \citep{gatys2016image}. The formulation of the problem in the vision domain is to transfer the visual style from one image to the other, as illustrated in Figure \ref{fig:style-transfer-vision} \footnote{Images sourced from \url{https://github.com/fzliu/style-transfer}}. Stylistic transfer in text is based on a similar premise, where, given a an arbitrary body of text and a predefined style governed by a set of attributes like sentiment, emotion, tense, authorship, a new body of text can be generated such that it incorporates all of the pre-defined attributes its generation is being conditioned on.

\begin{figure}[ht]
	\centering
	\includegraphics[width=.8\textwidth]{images/style-transfer-vision.png}
	\caption{\label{fig:style-transfer-vision}Sample of vision style transfer. Image (a) provides the content, image (b) provides the style and image (c) is the final generated image}
\end{figure}

This problem in the context of text was first introduced in 2012 \citep{xu2012paraphrasing} as a statistical model that attempted to paraphrase bodies of text in a different style using a simple replacement strategy. An few examples from this paper are shown in Table \ref{table:paraphrasing-for-style-results}. Since the overwhelming adopting of neural network based models in the language community, there have been many more works which break new ground in this area. These works will be described in detail in Section \ref{related-work-section}.

\begin{table}[ht]
	\centering
	\begin{tabular}{ | p{.45\linewidth} | p{.45\linewidth} | }
		\hline
		\textbf{Input}                                              & \textbf{Output}                                      \\
		\hline \hline
		i will bite thee by the ear for that jest .                 & i ’ ll bite you by the ear for that joke .           \\
		\hline
		what further woe conspires against mine age ?               & what ’ s true despair conspires against my old age ? \\
		\hline
		how doth my lady ?                                          & how is my lady ?                                     \\
		\hline
		hast thou slain tybalt ?                                    & have you killed tybalt ?                             \\
		\hline
		an i might live to see thee married once , i have my wish . & if i could live to see you married, i ’ ve my wish . \\
		\hline
		benvolio , who began this bloody fray ?                     & benvolio , who started this bloody fight itself ?    \\
		\hline
		what is your will ?                                         & what do you want ?                                   \\
		\hline
		call her forth to me .                                      & bring her out to me .                                \\
		\hline
	\end{tabular}
	\label{table:paraphrasing-for-style-results}
	\caption{Results of transferring authorship style from Shakespearan plays to modern english}
\end{table}

\section{Problem Statement}

The general problem statement the thesis tackles can be stated as exploratory foray into the previous approaches used for linguistic style transfer and also provides a novel approach to solve this open problem, and juxtaposes this approach and it's experimental results against the current state-of-the-art models.

%======================================================================
\chapter{Background}
%======================================================================

\section{Natural Language Generation}

Natural Language Generation (NLG) is a sub-field of Natural Language Processing that attempts to generate a sequence of words that resemble natural human languages. Traditionally, this can be done by either using production rules of a pre-defined grammar, or by performing statistical analyses of existing human-written texts to predict sequences of words based on their occurrence probabilities.

\section{Natural Language Understanding}

Natural Language Understanding (NLU) is another sub-field of Natural Language Processing which can be viewed as the converse of what is done in Natural Language Generation. Given a corpus of text, Natural Language Understanding is a collection of tasks that extract structured information from the text. A few examples of Natural Language Understanding tasks are sentiment analysis, emotion detection, entity-relation mapping, language comprehension etc.

\section{Recurrent Neural Networks}

Recurrent neural networks are a sub-class of artificial neural networks that can be considered a neural network equivalent to a Hidden Markov Model. Its units forms a directed graph that operates on a sequence of inputs when unrolled temporally. This makes them useful to extract features from arbitrary length sequences of input like audio or text.

\section{Sequence to Sequence Modeling}

This is a class of problems that models functions to map from one sequence to another. First introduced in \citep{sutskever2014sequence}, the general premise of sequence to sequence has been a flexible framework for modelling transformations made to arbitrary length sequences. In the natural language processing community, the main tasks that benefit from the encoder-decoder framework are neural machine translation, dialogue modeling, question answering for which there exists two distinct distributions of data, and the model is trained to learn the mapping from one to the other.

The paper defines a sequence-to-sequence model as on that learns the below function to map a sequence of inputs $x_1, ... , x_T$ to a sequence of outputs $y_1, ... , y_{T′}$, where the initial state $v$ is set to the hidden LSTM representation of $x_1, ... , x_T$.

\begin{equation}
	p(y_1, ... , y_{T′} | x_1, ... , x_T) =	\prod_{t=1}^T p(y_t | v, y_1, ... , y_{t−1})
\end{equation}


\section{Autoencoders}

Autoencoders are models that are parameterized to convert arbitrary data into a latent representation (encoder), and recover the original data back from the latent representation (decoder). In this setup, the degrees of freedom for the latent representation is usually much smaller than that of the actual data. A simple autoencoder architecture is depict in Figure \ref{fig:autoencoder-structure}. \footnote{Image sourced from https://commons.wikimedia.org/wiki/File:Autoencoder\_structure.png}

\begin{figure}[ht]
	\centering
	\includegraphics[width=.8\textwidth]{images/autoencoder-structure}
	\caption{\label{fig:autoencoder-structure} Schematic picture of an autoencoder architecture}
\end{figure}

By training a model to do this, two objectives can be achieved simultaneously:
\begin{itemize}
	\item The encoder part of the model could be used extract the most salient features of the data in a compressed representation, which is a friendlier format for downstream processing or learning algorithms. \citep{hinton2006reducing}
	\item The decoder part of the model could be used as a generator. Given that we can sample from the distribution of the existing latent representations learnt, or from a pre-defined prior (in a variational autoencoder), we can generate plausible novel data.
\end{itemize}

In the context of natural language the encoder can be utilized as a sentence-encoding feature-extractor and the decoder can be utilized as a generative model. Autoencoders are also used for being able to de-noise data given pairs of noisy and regular data, by learning a de-noising function. These properties in general make autoencoders a good framework to implement sequence-to-sequence models with.

%======================================================================
\chapter{Related Work} \label{related-work-section}
%======================================================================

\section{Controlling Linguistic Aspects of Text Generation}

In the past year, the work by \citep{hu2017toward} and \citep{ficler2017controlling} both expounded the applicability of linguistic style transfer. Both of these methods, as opposed to the historical used paraphrasing methods, utilized neural network models \citep{lecun2015deep}.

\citep{hu2017toward} use a variational autoencoder trained with the reconstruction objective and a KL-divergence minimization objective on the latent space with respect to a prior $p(z)$, as described in the original paper by \cite{kingma2013auto}.
\begin{eqnarray*}
	\mathcal{L}_{VAE}(\theta_G, \theta_E; x) &=&
	- \mathbb{E}_{q_E(z|x)q_D(c|x)}[log p_G(x|z,c)] \\ & &
	+ KL(q_E(z|x)||p(z))
\end{eqnarray*}

In addition to the reconstruction objective, the authors use additional discriminative training signals to adapt the desired attributes of the generated text.


%======================================================================
\chapter{Challenges}
%======================================================================

\section{Quantitative evaluation of language quality}

%======================================================================
\chapter{Methods}
%======================================================================

\section{Approach}

In this section, we describe our approach in detail. Our model is built upon an autoencoder with a sequence-to-sequence (Seq2Seq) neural network~\cite{sutskever2014sequence}, as described in Subsection~\ref{ss:seq2seq}. Then we introduce two auxiliary losses, multi-task loss and adversarial loss in Subsections~\ref{ss:multi} and \ref{ss:adv}, respectively. Subsection~\ref{ss:prediction} presents the approach to transfer style in natural language generation. Figure~\ref{fig:model-overview} depicts both training and prediction processes of our approach.


\begin{figure}[ht]
	\centering
	\includegraphics[width=0.9\linewidth]{images/model-overview.png}
	\caption{Overview of our approach.}
	\label{fig:model-overview}
\end{figure}

\subsection{Autoencoder}\label{ss:seq2seq}

An autoencoder encodes an input to a latent vector space, from which it decodes the input itself. By doing so, the autoencoder learns meaningful representations of data. This serves as our primary learning objective. Besides, we also use the autoencoder for text generation in the style-transfer application.

Let $\rmx=(x_1, x_2, \cdots x_n)$ be an input sentence. The encoder encodes $\rm x$ by a recurrent neural network (RNN) with gated recurrent units (GRU) \cite{cho2014learning}, and obtains a hidden state $\bm h$.

Then a decoder RNN generates a sentence, which ideally should be $\rmx$ itself. Suppose at a time step $t$, the decoder RNN predicts the word $x_t$ with probability $p(x_t|\bm h, x_1\cdots x_{t-1})$, then the autoencoder is trained with cross-entropy loss, given by
\begin{equation}\nonumber
	J_\text{rec}(\bm\theta_E,\bm\theta_D)= -\sum_{t=1}^n \log
	p(x_t|\bm h, x_1\cdots x_{t-1})
\end{equation}
where $\bm\theta_E$ and $\bm\theta_D$ are the parameters of the encoder and decoder, respectively.

Since this loss trains the autoencoder to reconstruct $\rmx$, it is also called \textit{reconstruction loss}.

Besides the above reconstruction loss, we design two auxiliary losses to disentangle the latent space $\bm h$. In particular, we hope that $\bm h$ can be separated into two spaces $\bm s$ and $\bm c$, representing style and content respectively, i.e., $\bm h = [\bm s ; \bm c]$, where $[\cdot;\cdot]$ denotes concatenation. This is accomplished by the below auxiliary losses.

\subsection{Multi-Task Loss} \label{ss:multi}

Our first auxiliary loss ensures the style space does contain style information. We build a classifier on the style space $\bm s$ predicting the style label $s$, which is a part of the training data.

This loss can be viewed as a \textit{multi-task} loss, which makes the neural network not only decode the sentence, but also predicts its sentiment. Similar multi-task losses are used in previous work for sequence-to-sequence learning \cite{luong2015multi}, sentence representation learning \cite{jernite2017discourse} and sentiment analysis \cite{balikas2017multitask}, among others.

In our application, we follow previous work \cite{hu2017toward,shen2017style,fu2017style} and treat the sentiment as the style of interest. We introduce a binary classifier
\begin{equation}
	p(s=1|\bm s;\bm\theta_\text{mult})=\sigma(\bm w_\text{mult}^\top \bm s + b_\text{mult})
\end{equation}
where $\bm\theta_\text{mult}=[\bm w_\text{mult}; b_\text{mult}]$ are the classifier's parameters for multi-task learning.
\begin{align} \label{eqn:Jmult}
	 & J_\text{mult}(\bm\theta_{E};\bm\theta_\text{mult})=                \\ \nonumber
	 & -s\log p_\text{mult}(s|\bm s) - (1-s)\log p_\text{mult}(1-s|\bm s)
\end{align}
where $\bm\theta_E$ are the encoder's parameters. (Notice that the bold letter $\bm s$ represents the encoded style vector, whereas the unbold letter $s$ represents the binary style label).


\subsection{Adversarial Learning} \label{ss:adv}

The above multi-task loss only operates on the style space, but does not have an effect on the content space $\bm c$.

We therefore apply an adversarial loss to disentangle the content space from style information, inspired by adversarial generation~\cite{goodfellow2014generative}, adversarial domain adaptation~\cite{liu2017adversarial}, and adversarial style transfer~\cite{fu2017style}.

The idea of adversarial loss is to introduce an adversary that deliberately discriminates style $s$ on the content vector $\bm c$. Then the autoencoder is trained to learn such a content vector space that its adversary cannot predict style information.

Concretely, the adversarial discriminator predicts style $s$ by a logistic regression
\begin{equation}
	p_\text{dis}(s=1|\bm c;\bm\theta_\text{adv})=\sigma(\bm w_\text{adv}^\top \bm c + b_\text{adv})
\end{equation}
where $\bm\theta_\text{dis}=[\bm w_\text{dis}; b_\text{dis}]$ are the parameters of the adversary. It is trained by
\begin{align}
	 & J_\text{dis}(\bm\theta_\text{dis})=                            \\ \nonumber
	 & -s\log p_\text{dis}(s|\bm c)-(1-s)\log p_\text{dis}(1-s|\bm c)
\end{align}
The adversarial loss appears similar to the multi-task loss as in Eq.~(\ref{eqn:Jmult}). However, it should be emphasized that, for the adversary, the gradient is not propagated back to the autoencoder, i.e., $\bm c$ is treated as shallow features.

Having trained an adversary, we would like the autoencoder to be tuned in such an \textit{ad hoc} fashion, that $\bm c$ is not discriminative in style. In other words, we penalize the entropy of the adversary's prediction, given by
\begin{equation}
	J_\text{adv}(\bm\theta_E)=\mathcal{H}(p_\text{dis}(s|\bm c))
\end{equation}
where $\mathcal{H}=-\sum_{i\in\text{labels}}p_i\log p_i$ is the entropy. The adversarial objective is maximized, in this phase, with respect to the encoder.



\subsection{Training Process}

To put it all together, our training process is a loop of the following processes:
\begin{itemize}
	\item minimize $J_\text{dis}(\bm\theta_\text{dis})$ w.r.t. $\bm\theta_\text{dis}$, and
	\item minimize $J_\text{rec}(\bm\theta_E, \bm\theta_D) + \lambda_\text{mult}(\bm\theta_E,\bm\theta_\text{mult}) -\lambda_\text{adv}
		      J_\text{adv}(\theta_E)$ w.r.t. $\bm\theta_E, \bm\theta_D, \bm\theta_\text{mult}$.
\end{itemize}
where $\lambda_\text{mult}$ and $\lambda_\text{adv}$ balance these losses.

We use the Adam optimizer \cite{kingma2014adam} with an initial learning rate of $10^{-3}$ and train the model for 50 epochs. Both the autoencoder and its adversary are trained once per epoch with $\lambda_\text{mult} = 1$ and $\lambda_\text{adv} = 0.3$.

\subsection{Generating Style-Transferred Sentences} \label{ss:prediction}

A direct application of our disentangled latent space is style transfer for natural language generation. For example, we can generate a sentence with generally the same meaning (content) but an opposite sentiment.

Let $\rmx_*$ be an input sentence with $\bm s_*$ and $\bm c_*$ being the encoded, disentangled style and content vectors, respectively. If we would like to transfer its content to a different style, we compute an empirical estimate of the target style's vector $\hat{\bm s}$ by
$\hat{\bm s}=\frac{\sum_{i\in\text{target style}}\bm s_i}{\text{\# target style samples}}$. The inferred target style $\hat{\bm s}$ is concatenated with the encoded content $\bm c_*$ for decoding (Figure~\ref{fig:model-overview}b).

\section{Experiments}
We conducted experiments on an Amazon product review dataset, following \cite{fu2017style}. It contains 131072, 2048, 32768 sentences for train, validation, and test, respectively, each sampling accompanied by binary sentiment labels.

\subsection{Disentangling Latent Space}


We first analyze how the style (sentiment) and content of the latent space are disentangled. We train a logistic classifier based on different latent spaces, and show results in Table~\ref{tab:classification}.

We see that the 128-dimensional content vector is not discriminative for style. It achieves an accuracy of 57\%, slightly better than random/majority guess. However, the 8-dimensional style vector $\bm s$, despite its low dimensionality, achieves significantly higher style classification accuracy. When combining content and style vectors, we achieve no further improvement. These results verify the effectiveness of our disentangling approach, because the style space does contain style information, whereas the content space doesn't.


\begin{table}
	\centering
	\begin{tabular}{lr}
		\hline
		Random/Majority guess           & 0.5000          \\ \hline
		Content latent space  ($\bm c$) & 0.5690          \\
		Style latent space ($\bm s$)    & \textbf{0.7817} \\
		Combined ($[\bm s;\bm c]$)      & 0.7815          \\
		\hline
	\end{tabular}
	\caption{Style classification accuracy.}
	\label{tab:classification}
\end{table}

\begin{figure}
	\includegraphics[width=\linewidth]{images/tsne-style-and-content}
	\caption{t-SNE plots of (a) style and (b) content spaces.}
	\label{fig:tsne}
\end{figure}


The latent space can be visualized with t-SNE plots~\cite{maaten2008visualizing} in Figure~\ref{fig:tsne}. As expected, sentences with different sentiments can be nicely separated in the style space (LHS), but are highly mixed in the content space (RHS).


\subsection{Style-Transfer Sentence Generation}

We apply the disentangled latent space to a style-transfer sentence generation task, where the goal is to generate a sentence with different sentiment (style). We followed \cite{fu2017style} and used two metrics: (1) For style transfer, we train a style classifier and predict the accuracy of the generated sentences. While the style classifier itself may not be perfect, it provides a quantitative way of evaluating the strength of style transfer. (2) For the content-preservation score, we compute a sentence embedding by min, max, and average pooling of word embeddings; then a cosine similarity is computed to evaluate how close two sentences are in meaning. Here, sentiment words from a stop list \cite{hu2004mining} are removed.

We compare our approach with state-of-the-art previous work in Table~\ref{tab:comparison-previous}. We re-conducted the experiments with their publicly available code on our data splits.
Results show that, our approach achieves a comparable content-preservation score with previous work, but a significantly better style-transfer score, showing that our disentangled latent space can be used for better style-transfer sentence generation.

Table~\ref{tab:ablation-results} presents the results of an ablation test. We see that both adversarial loss and reconstruction loss play a role in the strength of style transfer, and that they can be combined to further improve performance.

Some examples of style-transfer sentence generation are illustrated in Table~\ref{tab:transfer-samples}. We see that, with the empirically estimated style vector, we can flexibly control the sentiment of generated sentences.

\begin{table}[!t]
	\centering
	\begin{tabular}{l|rr}
		\hline
		\textbf{Model}                       & \textbf{Style Transfer} & \textbf{Content Preservation} \\
		\hline
		\hline
		Cross-alignment \cite{shen2017style} & 0.4609                  & 0.8830                        \\
		\hline
		Sentiment embed. \cite{fu2017style}  & 0.4009                  & 0.9246                        \\
		\hline
		Ours                                 & 0.7708                  & 0.8958                        \\
		\hline
	\end{tabular}
	\caption{Comparison with previous approaches.}
	\label{tab:comparison-previous}
\end{table}

\begin{table}[!t]
	\centering
	\begin{tabular}{l|rr}
		\hline
		\textbf{Training Loss}                          & \textbf{Style Transfer} & \textbf{Content Preservation} \\
		\hline
		\hline
		$J_\text{rec}$                                  & 0.5053                  & 0.9103                        \\
		\hline
		$J_\text{rec}$, $J_\text{adv}$                  & 0.5901                  & 0.9121                        \\
		\hline
		$J_\text{rec}$, $J_\text{mult}$                 & 0.6445                  & 0.9053                        \\
		\hline
		$J_\text{rec}$, $J_\text{adv}$, $J_\text{mult}$ & 0.7708                  & 0.8958                        \\
		\hline
	\end{tabular}
	\caption{Ablation test.}
	\label{tab:ablation-results}
\end{table}

\begin{table}[!t]
	\centering
	\resizebox{\linewidth}{!}{
		\begin{tabular}{| p{0.5\linewidth} | p{0.5\linewidth} |}
			\hline
			\textbf{{Original}}                                                        & \textbf{Transferred (Positive $\rightarrow$ Negative)}                    \\
			\hline
			\hline
			i bought this cuisipro mister to replace my old mister last june-(number)  & i bought this a couple of times and was disappointed in this product      \\
			\hline
			quality is good, recevied in time and works as expected.                   & quality is good but i am returning it                                     \\
			\hline
			all in all, i am very happy with this headset.                             & all in all i was expecting a good product                                 \\
			\hline
			\hline
			\textbf{{Original}}                                                        & \textbf{Transferred (Negative $\rightarrow$ Positive)}                    \\
			\hline
			\hline
			i sent it back and requested a refund, and never got the refund.           & i sent it back and gave it a try and it works great                       \\
			\hline
			so i tried just one (number) piece each n still the same results.          & so i bought the two sizes and the other ones are great                    \\
			\hline
			i am going to buy a replacement and wish i had sent this back for a refund & i am going to go through the same time and i have been using it for years \\
			\hline
		\end{tabular}}
	\caption{Examples of style-transfer generation.}
	\label{tab:transfer-samples}
\end{table}


%======================================================================
\chapter{Observations}
%======================================================================

\section{Evaluation Metrics}

\subsection{Cycle Loss}
Cycle loss is not a good metric because while transferring from $X$ to $Y$, and back to $X'$, the variable $Y$ is not observable and if the neural network function is accidentally fitted to transform the content domain, this metric will not be able to tell the difference between an overfitted model and a good model.


%----------------------------------------------------------------------
% END MATERIAL
%----------------------------------------------------------------------

% B I B L I O G R A P H Y
% -----------------------

% The following statement selects the style to use for references.  It controls the sort order of the entries in the bibliography and also the formatting for the in-text labels.
\bibliographystyle{unsrtnat}
% This specifies the location of the file containing the bibliographic information.  
% It assumes you're using BibTeX (if not, why not?).
\cleardoublepage % This is needed if the book class is used, to place the anchor in the correct page,
% because the bibliography will start on its own page.
% Use \clearpage instead if the document class uses the "oneside" argument
\phantomsection  % With hyperref package, enables hyperlinking from the table of contents to bibliography             
% The following statement causes the title "References" to be used for the bibliography section:
\renewcommand*{\bibname}{References}

% Add the References to the Table of Contents
\addcontentsline{toc}{chapter}{\textbf{References}}

\bibliography{uw-ethesis}
% Tip 5: You can create multiple .bib files to organize your references. 
% Just list them all in the \bibliogaphy command, separated by commas (no spaces).

% The following statement causes the specified references to be added to the bibliography% even if they were not 
% cited in the text. The asterisk is a wildcard that causes all entries in the bibliographic database to be included (optional).
\nocite{*}

\end{document}
